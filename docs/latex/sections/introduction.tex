\section{Introduction}

This is the help document for the Urban Radio Channel (URC), developed by Craig Cooper and Abhinay Mukunthan. It has been released under the GLGPL.

This documents contains instructions on how to use the model in VEINS \cite{sommer_veins_2010}, a simulation framework that combines OMNeT++ \cite{varga_omnet++_2009} with SUMO \cite{krajzewicz_sumo_2011} via the TraCI communications protocol \cite{wegener_traci:_2008}. The modules are built upon MiXiM \cite{viklund_mixim_2011}. To build URC, VEINS 2.0 or later is required.

The model combines calculations to account for three separate elements of the channel:
\begin{enumerate}
 \item Path Loss - Accounted for using the CORNER model proposed by Giordano et al \cite{giordano_corner:_2011}. This uses the layout of an urban environment to determine whether a destination node is within Line-Of-Sight of a transmitting vehicle. Different path loss calculations are applied based on this classification. The model was verified and improved upon by Mukunthan el al \cite{mukunthan_experimental_2013}.
 \item Fading - An environmentally dependant fading model was presented in \cite{cooper_dynamic_2014}, which uses the building layout to estimate the multi-path components and estimate the parameters of a Ricean fading model. A pre-simulation step to compute the fading parameters is required.
 \item Vehicular obstructions - A deterministic model based on Knife-Edge shadowing was developed and verified experimentally by Wang et al \cite{wang_influence_2009}. This was selected to account for the transient nature of vehicular obstructions.
\end{enumerate}

