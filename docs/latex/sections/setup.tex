\section{Setup}

\subsection{Building the library}
To build URC for OMNeT++, open a terminal and navigate to where you unzipped the URC folder. Type the following command:
\begin{lstlisting}[frame=single]
 make install_OMNETPP 
  OMNETPP_INSTALL_DIR=<install_dir>
  VEINS_ROOT=<veins_dir>
\end{lstlisting}
This will compile the URC library, and copy the headers and library files to the given installation directory. The directories provided must be relative directories.

If you want to build in debug mode, use 
\begin{lstlisting}[frame=single]
 make install_OMNETPP DEBUGMODE=1
  OMNETPP_INSTALL_DIR=<install_dir>
  VEINS_ROOT=<veins_dir>
\end{lstlisting}

Note, you should make sure you've built VEINS before building URC, otherwise you'll get some annoying errors about cpp and cc files in OMNeT++.

\subsection{Building the utilities}

URC has several utilities for preparing data for the engine. Most of these are written in Python (See Section \ref{subsect:sumo2corner}), but two are written in C++ and need to be built: \begin{enumerate}
 \item \textit{BuildingSolver} - a tool that constructs an outline of buildings from a SUMO map.
 \item \textit{Raytracer} - a tool that performs the $K$-factor approximation method in \cite{cooper_dynamic_2014} to a given SUMO map.
\end{enumerate}

To build these, type:
\begin{lstlisting}[frame=single]
 make BuildingSolver
\end{lstlisting}
or
\begin{lstlisting}[frame=single]
 make Raytracer
\end{lstlisting}

The binary will be stored in the \textit{bin} directory of the URC root folder.
