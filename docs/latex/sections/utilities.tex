\section{Utilities}

URC has a number of utilities that are used to prepare road data for simulations.

\subsection{Sumo2Corner}\label{subsect:sumo2corner}

Our CORNER implementation is different from that of UCLA \cite{giordano_corner:_2011}. Their approach used geometric distance calculations to determine the link on which a transmitter and receiver were travelling, and then classified the Line-Of-Sight (LOS) on the fly. This method is not computationally efficient for very large simulations, unfortunately. We have leveraged the unique features of VEINS, which allows us to simply fetch the current road IDs from SUMO. The LOS state can then be indexed in a database of link-pairs and corresponding classifications. By matching the textual Road ID to a numerical ID, the model is made more efficient. This necessitates a presimulation step of analysing the road network and analysing the classifications between all links in the network.

The script \textit{Sumo2Corner.py} contains all the preprocessing functions for preparing a map to work with the CORNER components of URC. It can operate on either a SUMO network file, or an OSM file.

The script accepts a valid filename, analyses the road network, and generates five files:
\begin{enumerate}
 \item Link Lookup - A file of road links and numerical IDs of their corresponding nodes. Number of lanes is also included here.
 \item Node Lookup - A file of road intersections and their geographical locations within the map.
 \item Classification database - This database contains a list of source-destination road links, their LOS state (LOS, NLOS1, or NLOS2), and additional street information required for the pathloss calculation.
 \item Link-Name Lookup - The classification database stores links and nodes as numerical indices for performance consideration. However, certain functions like the $K$-factor lookup relies on knowing the ASCII name of the link the current vehicle is occupying, as well as its lane. Thus, this lookup table allows URC to find the ID of a link based on its name.
 \item Internal Link Lookup - SUMO includes internal links, which connect lanes across intersections. If a map is created with internal links, an additional lookup is required to match an internal link to the intersection the car is crossing. This will allow the CORNER functionality to determine an LOS state for cars crossing intersections.
\end{enumerate}

\textit{Sumo2Corner.py} may also work on OSM maps. When given an OSM map, it extracts the road geometry into a SUMO net file, while also extracting the building geometry data for use with the Raytracer (See Section \ref{subsect:raytracer}). The script may be given the option of stripping out paths such as walkways and bicycle tracks, which the SUMO converter mistakes for vehicular roads.

To use with a SUMO network file, use the command:
\begin{lstlisting}[frame=single]
 Sumo2Corner.py -n <netfile>
\end{lstlisting}
This will use the base filename, sans extension, to name the files it generates.

To use with an OSM map, use the command:
\begin{lstlisting}[frame=single]
 Sumo2Corner.py -o <osmfile>
\end{lstlisting}
To strip walkways from the OSM file before conversion, use the \textbf{-s} option.

Note, if you specify your own SUMO map, you won't be able to get building geometry using this script. To generate building geometry, use the \textit{BuildingSolver} tool (See Section \ref{subsect:buildingsolver}).

\subsection{BuildingSolver}\label{subsect:buildingsolver}

\textit{Sumo2Corner.py} can generate building geometry data from an OSM map, but not from a SUMO network file. The \textit{BuildingSolver} tool traces around the road network and generates building outlines in between roads. This data can then be fed into the Raytracer (See Section \ref{subsect:raytracer}) for $K$-factor calculation.

To use program, type the following command:
\begin{lstlisting}[frame=single]
 BuildingSolver <basefile>
   <lanewidth> <footPathWidth>
\end{lstlisting}
This script will load the CORNER files generated in the previous section (specified by \textbf{basefile}) and generate a set of buildings. The filename will be \textbf{basefile.corner.bld}.

\subsection{Raytracer}\label{subsect:raytracer}

The Raytracer is significantly more complicated than the others. As this program can be computationally intensive, we have attempted to write it in a manner conducive to parallel processing. To that end, the program first generates configurations, dividing the map area into several smaller segments. This allows multiple instances of the program to be run on different areas of the map simultaneously.

Note: This technically isn't a raytracer, but a ray-launcher. 

To begin, run \textit{Raytracer} with the \textbf{-g} option, followed by these configuration tags:
\begin{itemize}
 \item \textbf{-b} - Specify base filename of CORNER files as generated by \textit{Sumo2Corner}. Manditory.
 \item \textbf{-r} - Specify a ray count. Default: 256
 \item \textbf{-G} - Receiver Gain. Default: 1
 \item \textbf{-i} - This is the number of metres between consecutive positions for $K$-factor approximation. Default: 10
 \item \textbf{-c} - Number of cores to use in tracing the rays. Default: 2
 \item \textbf{-N} - Number of areas to divide the map into. Default: 1
 \item \textbf{-l} - Road width in metres. Default: 5
 \item \textbf{-F} - Filename to write configurations into. Default: config
 \item \textbf{-V} - Visualise the raytracing in progress. See Section \ref{subsubsect:visualise}
\end{itemize}

Once the configurations have been generated, run the program like so:
\begin{lstlisting}[frame=single]
 Raytracer <config> <run_number>
\end{lstlisting}

If you have a script or system that allows you to run a program with multiple configurations (such as a simulation cluster program), you can adapt your system to distribute \textit{Raytracer} over multiple computers, to be run in parallel. When the entire run is complete, you will be presented with a file named \textbf{basename-run\#.urc.k}. The basename will be the one specified in the configuration. These files will need to be combined into one file. At present, such a functionality has yet to be programmed.

Each file begins with the increment value specified in the file, followed by the number of source links contained within. The data is then organised in order:
\begin{enumerate}
 \item Source Link ID
 \item Increment along Source Link
 \item Source Lane
 \item Destination Link ID
 \item Increment along Destination Link
 \item Destination Lane
 \item $K$-factor
\end{enumerate}
When combining, ensure that the database is sorted according to Source Link ID.

\subsubsection{Visualiser} \label{subsubsect:visualise}
It is possible to visualise the Raytracer program in progress. The program must be built for this first, using the command:\begin{lstlisting}[frame=single]
 make Raytracer USE_VISUALISER=1
\end{lstlisting}
This uses Allegro 5.0 \cite{hargraves_allegro_2014}, a cross-platform multimedia library, to display the raytracer functioning. In order to build the visualiser, you will need a working installation of this library.

\subsection{vehicleTypes.py}

The script \textit{vehicleTypes.py} allows you to assign specified vehicle types to cars in an existing SUMO route definition file. The script is executed with the command:
\begin{lstlisting}[frame=single]
 vehicleTypes.py <car_def> <in> <out>
\end{lstlisting}
The car definition file is a csv file. Each line has the format:
\begin{enumerate}
 \item \textbf{id} - Name of the car type.
 \item \textbf{accel} - Acceleration coefficient.
 \item \textbf{decel} - Deceleration coefficient.
 \item \textbf{sigma} - Driver imperfection.
 \item \textbf{color} - Colour in SUMO GUI interface.
 \item \textbf{length} - Length of car.
 \item \textbf{width} - Width of car.
 \item \textbf{height} - Height of car.
 \item \textbf{weight} - Probability of this car being selected at random $(0,1)$. 
\end{enumerate}
The first six parameters are explained in more detail on the SUMO wiki. The dimensions of the car are used in the shadowing model. Note that the total weight of all the cars in the file must sum to 1. An example vehicle definition file is in \textit{data/vTypes.csv}.

